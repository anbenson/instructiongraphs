\documentclass[12pt]{article}

\usepackage{amsmath}

\title{Instruction Graph Grammar}
\author{Andrew Benson}
\date{}

\newcommand{\nonterm}[1]{\langle\text{#1}\rangle}
\newcommand{\term}[1]{\textbf{#1}}

\begin{document}
\maketitle

\begin{align*}
  \nonterm{program}\quad ::&=\quad \term{$\emptyset$}\quad
                            |\quad \term{P}(\nonterm{vertices},\ \term{int})\\
  \nonterm{vertices}\quad ::&=\quad \term{S}(\nonterm{vertex})\quad
                             |\quad \term{Cons}(\nonterm{vertex},\
                                                \nonterm{vertices})\\
  \nonterm{vertex}\quad ::&=\quad \term{V}(\term{int},\ \nonterm{content})\\
  \nonterm{content}\quad ::&=\quad
                      \term{Do}(\nonterm{action},\ \term{int})\quad\\
             &|\qquad \term{DoU}(\nonterm{action},\
                                     \nonterm{cond},\ \term{int})\\
             &|\qquad \term{Cond}(\nonterm{cond},\ \term{int},\ \term{int})\\
             &|\qquad \term{GoTo}(\term{int})\\
             &|\qquad \term{End}
\end{align*}

An \term{int} is any integer.\\

An $\nonterm{action}$ represents the kinds of actions (like movement) a specific
robot may be able to perform. We assume a grammar defining $\nonterm{action}$
exists.\\

A $\nonterm{cond}$ represents the kinds of conditions (like whether an object is
some distance ahead) a specific robot may be able to sense. We assume a grammar
defining $\nonterm{cond}$ exists.\\

\end{document}
