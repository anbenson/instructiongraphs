\documentclass[12pt]{article}

\usepackage{amsmath}
\usepackage{amssymb}
\usepackage{mathpartir}
\usepackage{fancyhdr}
\usepackage{ig}

\fancypagestyle{simple}
{
  \fancyhf{}
  \cfoot{Dynamics Page \thepage}
  \renewcommand{\headrulewidth}{0pt}
}
\pagestyle{simple}

\title{Instruction Graph Dynamics}
\author{Andrew Benson}
\date{}

\begin{document}
\maketitle

\thispagestyle{simple}
\section{Configuration}

To describe a state midway through execution, we define
\begin{alignat*}{4}
  \sort{Configuration}\quad
    &\nterm{cfg}\quad & &::=\quad &
      &\cfg{\nterm{n}}{\nterm{vs}}{\nterm{I}}{\nterm{O}}\qquad &
        &\text{configurations}
\end{alignat*}

where $\nterm{n} \in \mathbb{Z}$, the integers, $\nterm{vs} \in
\sort{Vertices}$, $\nterm{I}$ is a $bool$ list, representing the input used to
satisfy a $\sort{Condition}$ $\nterm{cnd}$, and $\nterm{O}$ is an
$\sort{Action}$ list, representing the ordered (but reversed) list of actions
that are executed.\\

\section{Terminated}

$\terminated{\cfg{\nterm{n}}{\nterm{vs}}{\nterm{I}}{\nterm{O}}}$ means that the
state with vertex represented by $\nterm{n}$ in vertices $\nterm{vs}$ with
remaining input $\nterm{I}$ and current output $\nterm{O}$ is in a finished
state for the program execution context.

\begin{mathpar}
  \inferrule{
    \ver{\nterm{n}}{\progend} \in \nterm{vs}
  }{
    \terminated{\cfg{\nterm{n}}{\nterm{vs}}{\nterm{I}}{\nterm{O}}}
  }
\end{mathpar}

\section{Waiting}

$\waiting{\cfg{\nterm{n}}{\nterm{vs}}{\nterm{I}}{\nterm{O}}}$ means that the
state with vertex represented by $\nterm{n}$ in vertices $\nterm{vs}$ with
remaining input $\nterm{I}$ and current output $\nterm{O}$ cannot proceed, as it
requires more input to continue.

\begin{mathpar}
  \inferrule{
    \ver{\nterm{n}}{\dountil{\nterm{a}}{\nterm{cnd}}{\nterm{n'}}} \in \nterm{vs}
  }{
    \waiting{\cfg{\nterm{n}}{\nterm{vs}}{\nterm{[\ ]}}{\nterm{O}}}
  }

  \inferrule{
    \ver{\nterm{n}}{\ifelse{\nterm{cnd}}{\nterm{n'}}{\nterm{n''}}} \in \nterm{vs}
  }{
    \waiting{\cfg{\nterm{n}}{\nterm{vs}}{\nterm{[\ ]}}{\nterm{O}}}
  }
\end{mathpar}

\section{Steps}

$\steps{
  \cfg{\nterm{n}}{\nterm{vs}}{\nterm{I}}{\nterm{O}}
}{
  \cfg{\nterm{n'}}{\nterm{vs}}{\nterm{I'}}{\nterm{O'}}
}$ means that the state with vertex represented by $\nterm{n}$ in vertices
$\nterm{vs}$ with remaining input $\nterm{I}$ and current output $\nterm{O}$
continues to the state with vertex represented by $\nterm{n'}$ in vertices
$\nterm{vs}$ with remaining input $\nterm{I'}$ and current output $\nterm{O'}$.

\begin{mathpar}
  \inferrule{
    \ver{\nterm{n}}{\doonce{\nterm{a}}{\nterm{n'}}} \in \nterm{vs}
  }{
    \steps{
      \cfg{\nterm{n}}{\nterm{vs}}{\nterm{I}}{\nterm{O}}
    }{
      \cfg{\nterm{n'}}{\nterm{vs}}{\nterm{I}}{\cons{\nterm{a}}{\nterm{O}}}
    }
  }

  \inferrule{
    \ver{\nterm{n}}{\dountil{\nterm{a}}{\nterm{cnd}}{\nterm{n'}}} \in \nterm{vs}
  }{
    \steps{
      \cfg{\nterm{n}}{\nterm{vs}}{\cons{\nterm{true}}{\nterm{I}}}{\nterm{O}}
    }{
      \cfg{\nterm{n'}}{\nterm{vs}}{\nterm{I}}{\cons{\nterm{a}}{\nterm{O}}}
    }
  }

  \inferrule{
    \ver{\nterm{n}}{\dountil{\nterm{a}}{\nterm{cnd}}{\nterm{n'}}} \in \nterm{vs}
  }{
    \steps{
      \cfg{\nterm{n}}{\nterm{vs}}{\cons{\nterm{false}}{\nterm{I}}}{\nterm{O}}
    }{
      \cfg{\nterm{n}}{\nterm{vs}}{\nterm{I}}{\cons{\nterm{a}}{\nterm{O}}}
    }
  }

  \inferrule{
    \ver{\nterm{n}}{\ifelse{\nterm{cnd}}{\nterm{n'}}{\nterm{n''}}} \in \nterm{vs}
  }{
    \steps{
      \cfg{\nterm{n}}{\nterm{vs}}{\cons{\nterm{true}}{\nterm{I}}}{\nterm{O}}
    }{
      \cfg{\nterm{n'}}{\nterm{vs}}{\nterm{I}}{\nterm{O}}
    }
  }

  \inferrule{
    \ver{\nterm{n}}{\ifelse{\nterm{cnd}}{\nterm{n'}}{\nterm{n''}}} \in \nterm{vs}
  }{
    \steps{
      \cfg{\nterm{n}}{\nterm{vs}}{\cons{\nterm{false}}{\nterm{I}}}{\nterm{O}}
    }{
      \cfg{\nterm{n''}}{\nterm{vs}}{\nterm{I}}{\nterm{O}}
    }
  }

  \inferrule{
    \ver{\nterm{n}}{\goto{\nterm{n'}}} \in \nterm{vs}
  }{
    \steps{
      \cfg{\nterm{n}}{\nterm{vs}}{\nterm{I}}{\nterm{O}}
    }{
      \cfg{\nterm{n'}}{\nterm{vs}}{\nterm{I}}{\nterm{O}}
    }
  }
\end{mathpar}

\end{document}
